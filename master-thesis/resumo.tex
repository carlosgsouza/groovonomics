A recente populariza��o de linguagens dinamicamente tipadas, como Ruby e JavaScript, tem chamado a aten��o para a discuss�o sobre os impactos de diferentes sistemas de tipos sobre o desenvolvimento de software.
Tipagem est�tica permite que o compilador encontre erros de tipos mais cedo e potencialmente melhora a legibilidade e manutenibilidade do c�digo.
Por outro lado, c�digo "n�o tipado" pode ser mais f�cil de se alterar e requer menos trabalho dos programdores.
Esta disserta��o tenta identificar qual � o ponto de vista dos programdores sobre esses compromissos.
Uma an�lise do c�digo fonte de 6638 projetos escritos em Groovy, uma linguagem de programa��o com tipagem opcional, mostra em que cen�rios programadores preferem tipar ou n�o suas declara��es.
Nossos resultados mostram que tipos s�o populares na defini��o da interface de m�dulos, mas s�o menos usados em scripts, classes de teste e c�digo frequentemente alterado.
N�o h� correla��o entre o tamanho e a idade de projetos e como estes s�o tipados.
Por fim, tamb�m foi poss�vel encontrar evid�ncias de que a experi�ncia de programdores com outras linguagens de programa��o possui influ�ncia sobre como tipos s�o usados por esses programadores.

\keywords{Sistemas de Tipos, An�lise de Reposit�rios, Tipagem Opcional, Groovy}
